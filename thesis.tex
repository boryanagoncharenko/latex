\documentclass[parskip=full]{uvamscse}

\usepackage[utf8x]{inputenc}
\usepackage[T1]{fontenc}
\usepackage[toc,page]{appendix}
\usepackage{pgf-pie}
\renewcommand{\chapterautorefname}{Chapter}

\setcounter{secnumdepth}{4}

\titleformat{\paragraph}
{\normalfont\normalsize\bfseries}{\theparagraph}{1em}{}
\titlespacing*{\paragraph}
{0pt}{3.25ex plus 1ex minus .2ex}{1.5ex plus .2ex}

\input{program-listings}
\newcommand{\cmd}[1]{\texttt{$\backslash$#1}}

\title{Detecting Violations of CSS Code Conventions}
% \coverpic[100pt]{figures/terminal.png}
% \subtitle{}
% \date{Spring 2014}


\author{Boryana Goncharenko}
\authemail{boryana.goncharenko@gmail.com}

\abstract{    

Code conventions are meant to preserve code base consistency and express
preference of a particular programming style. Often, code conventions are
expressed in natural language and it is a responsibility of the developers to
read, understand and apply them. Typically, developers need to ensure that
their code complies to a given style guide manually. There are a number of
tools that can automatically detect violations of conventions. However,
current solutions remain rigid, laborious or with limited scope. The solution
presented in this thesis solves this problem. It consists of a domain--specific 
language that expresses existing CSS conventions and its interpreter
that finds violations automatically.

First, the need for CSS conventions is evaluated based on whether CSS is still
handcrafted. Second, existing CSS code conventions are discovered and
analyzed. Third, a domain-specific language capable of expressing existing CSS
code conventions is designed. The thesis contains a specification of the
designed domain--specific language and an implementation of its interpreter
that detects violations automatically.

}


\begin{document}
\maketitle

% !TEX root = thesis.tex
\chapter{Introduction}

Code conventions put constraints on how code should be written in the context
of a project, organization or programming language. Style guides can comprise
conventions that refer to whitespacing, indentation, code layout, preference
of syntactic structures, code patterns and antipatterns. They are mainly used
to achieve code consistency, which in turn improves the readability,
understandability and maintainability of the
code~\cite{Buse2010,tenny1988program,smit2011code}.

Style guides are often designed in an ad hoc manner. Coding conventions typically live in documents
that contain a description of each rule in natural language accompanied by code examples. This is
the case with the style guidelines of Mozilla~\cite{Mozilla}, Google~\cite{Google},
GitHub~\cite{GitHub}, WordPress~\cite{Wordpress} and Drupal~\cite{Drupal}. It is the responsibility
of the developers to ensure that their code complies to a given style guide. Typically, they need to
read and understand the conventions and then apply them manually. Such an approach introduces a
number of issues. First, using natural language can make guidelines incorrect, ambiguous, implicit
or too general. Second, the fact that developers apply conventions manually increases the chances of
introducing violations accidentaly. There are a number of tools that can automatically detect
violations of conventions, however, current solutions are often hard to customize or are limited to
one type of violations, e.g. only whitespacing.

The core idea behind the project is to provide a solution that lets developers express an arbitrary
set of coding conventions and detect their violations automatically. Writing conventions
in an executable form could assist authors in detecting incorrect, ambiguous or inconsistent
guidelines. Automatic detection of violations could minimize the effort required by developers to
write code that complies to the guidelines. To meet the constraints of a Master’s project, the
implementation is limited to the domain of Cascading Style Sheets (CSS). The project requires
determining the need for CSS code conventions in organizations, collecting and analyzing available
style guides, and providing a way to express conventions. Specifically, the project attempts to
answer the following set of questions:

  \begin{description}
    \item \textbf{Research Question 1} Do developers still maintain plain CSS?
    \item \textbf{Research Question 2} What code conventions for CSS exist?
    \item \textbf{Research Question 3} How to express existing CSS code conventions?
  \end{description}

The thesis is organized as follows. \autoref{sec:background} provides
information about previous studies and defines concepts and terms used
throughout the thesis. \autoref{sec:evaluating} presents the research approach
used to determine whether CSS is handcrafted and analysis of the gathered
results. \autoref{sec:discovering} contains the research method  used to
discover existing code conventions and the results of the research. The design
and validation of the DSL are presented in \autoref{sec:expressing}.
\autoref{sec:conclusion} concludes the thesis.


% !TEX root = thesis.tex
\chapter{Background}
\label{sec:background}

This chapter provides background information for concepts used in the thesis.
First, a brief definition of ontologies is presented and the main constructs
of the Bunge-Wand-Weber are listed. Second, the concept of ontological
analysis is introduced and the main types of ontological discrepancies are
defined.

\section{Bunge-Wand-Weber ontology}

\begin{table}[b]
\caption{Selected ontological constructs in the BWW representation model
\label{tab:bwwmodel}}
\begin{center}
\begin{longtable}{ | p{11em} | p{30em} | }
\hline
 &  \\
\textbf{Ontological construct} & \textbf{Explanation} \\
 &  \\
\hline
Thing & The elementary unit in the BWW ontological model. The real world is made up of things. A composite thing may be made up of other things (composite or primitive).~\cite{wand1995deep} \\ 
\hline
Properties & \multirow{7}{30em}{Things possess properties. A property is modeled via a function that maps the thing into some value. A property of a composite thing that belongs to a component thing is called an hereditary property. Otherwise it is called an emergent property. A property that is inherently a property of an individual thing is called an intrinsic property. A property that is meaningful only in the context of two or mode things is called a mutual or relational property.~\cite{wand1995deep}} \\ 
 &  \\ 
 &  \\
 &  \\
 &  \\
 &  \\
 &  \\
 \hline
 State & A vector of values for all property functions of a thing.~\cite{wand1995deep} \\
 \hline
 Event & A change of state of a thing. It is affected via a transformation.~\cite{wand1995deep} \\
 \hline
 Transformation & A mapping from a domain comprising states to a codomain comprising states.~\cite{wand1995deep} \\
 \hline
 History & The chronologically ordered states that a thing traverses.~\cite{weber1996analytical} \\
 \hline
 Coupling & A thing acts on another thing if its existence affects the history of the other thing. The two things are said to be coupled or interact.~\cite{wand1995deep} \\
 \hline
 Class & A class is a set of things that can be defined via their possessing a characteristic property.~\cite{weber1996analytical} \\ 
 \hline
 Subclass & A set of things that can be defined via their possessing the set of properties in
a class plus an additional set of properties.~\cite{weber1996analytical} \\ 
\hline
 System & A set of things is a system if, for any bi-partitioning of the set, couplings exist among things in the two subsets.~\cite{wand1995deep} \\
 \hline
 System Composition & The things in the system are its composition.~\cite{wand1995deep} \\
 \hline
 System Environment &  Things that are not in the system but interact with things in the system are called the environment of the system.~\cite{wand1995deep} \\
 \hline
\end{longtable}
\end{center}
\end{table}
A \textbf{conceptualization} is an abstract, simplified view of the world that is represented for
some purpose~\cite{gruber1995toward}. It consists of the concepts that are assumed to exist in some
area of interest and their relationships~\cite{gruber1995toward}. An \textbf{ontology} is an
explicit specification of a conceptualization~\cite{gruber1995toward}. It describes what is
fundamental in the totality of what exists and it defines the most general categories to which we
need to refer in constructing a description of reality~\cite{milton2004top}.

Researchers distinguish between different kinds of ontologies. For example,
based on the specificity of their constructs ontologies can be top-level and
domain-specific~\cite{milton2004top}. Ontologies of the former type are
highly general and provide the theoretical foundations for representation and
modeling of systems. Ontologies of the latter type define concepts and their
relations only for a particular domain. A domain-specific ontology is based on
a specific top-level ontology if it uses the categories defined by the high
level ontology~\cite{milton2004top}.

The Bunge-Wand-Weber (BWW) ontology~\cite{wand1990ontological} is a high-level ontology used in the
representation model developed by Wand and Weber~\cite{wand1995deep}. \autoref{tab:bwwmodel}
presents a selected set of the ontological constructs in the BWW ontology.

\section{Ontological analysis}

\textbf{Ontological analysis} is an established approach for evaluating the quality of software engineering notations~\cite{moody2009physics}. It consists of a two way comparison between a set of modeling grammar constructs and a set of ontological constructs. The \textbf{interpretation mapping} compares the notation with the ontology and the \textbf{representation mapping} compares the ontology with the notation~\cite{gehlert2007toward}. The underpinning of ontological analysis is that modeling grammars are incomplete if they are not able to represent what exists in reality~\cite{green2000integrated}. Furthermore, the analysis requires one-to-one mapping between the modeling grammar and the ontological constructs. Any deviation from such correspondence leads to a discrepancy (\autoref{fig:ontoanalysis}).

\begin{figure}[b]
  \centering
  \caption{Ontological Analysis~\cite[p.92]{Weber1997}}
  \label{fig:ontoanalysis}
  \includegraphics[width=0.6\textwidth]{ontoanalysis.pdf}
\end{figure}

\textbf{Construct deficit} occurs when an ontological construct does not have a corresponding construct in the modeling grammar. \textbf{Construct redundancy} is observed when a single ontological construct maps to more than one modeling grammar construct. \textbf{Construct overload} appears when a modeling grammar construct corresponds to more than one ontological construct. \textbf{Construct excess} emerges when a modeling grammar construct does not map to any ontological construct.~\cite{moody2009physics}


\chapter{Evaluating the Need for CSS Code Conventions}
\label{sec:evaluating}

\section{Research Method}

Despite the new features added in the second~\cite{CSS2} and third~\cite{CSS3} versions of CSS, the
language has obvious limitations, for example, lack of variables. A number of preprocessors have
evolved to tackle the downsides of CSS. Solutions such as SASS~\cite{SASS}, LESS~\cite{LESS} and
Stylus~\cite{Stylus} offer enhanced or even different syntax and translate it to CSS. Preprocessors
are not only ubiquitously recommended, but also widely adopted in practice. The presence of such
solutions poses the question whether conventions for CSS are required at all. If nowadays CSS is
generated and not maintained, the need for CSS conventions is substituted with need for preprocessor conventions.

To determine whether CSS is still handcrafted, all commits to open source repositories hosted on
GitHub for the period Jan-Apr 2015 were analyzed. To differentiate between plain CSS and
preprocessor code, the extensions of all files in the commits were inspected. In case the commit
contains a file with extension \texttt{.scss}, \texttt{.sass}, \texttt{.less} or \texttt{.styl}, it
is considered preprocessor maintenance. In case the commit contains files with the \texttt{.css}
extension and no preprocessor extensions, it is considered maintenance of plain CSS. Since the main
objective of the search is finding maintained files, only files that have been modified are taken
into consideration. Files that have been added are excluded from the results, since developers often
add third-party CSS libraries to their repositories.


\section{Results}

A total of 1,589,713 commits to 1,311,654 public repositories have been analyzed. X of these were
not processed due to errors. Typical errors contain repositories that have been turned private
and... The number of commits that maintain any form of CSS is X. Figure \ref{cssresults} summarizes
the findings.

\begin{figure}[h!]
  \centering
  \caption{Results}
  \label{cssresults}
  \includegraphics[width=0.5\textwidth]{cssresults}
\end{figure}

\section{Analysis}

There are a number of limitations that need to be considered before interpreting the results of the
conducted research. Firstly, the search is conducted on a single hosting platform - GitHub. That
said, currently GitHub reports having over 10 million users and 24 million
repositories~\cite{GitHub}, which makes it the largest code host in the
world~\cite{gousios2014lean}. Secondly, the search is narrowed to the publicly available
repositories. Thus, it lies on the premise that there is not a significant difference between the
public and private repositories hosted on the platform. Thirdly, the search detects only the four
most popular preprocessor extensions and omits other preprocessors. It is possible that a number of
custom preprocessors are used in practice. However, it is assumed that the number of such commits
would not increase the number of total CSS commits to the extend at which the percentage of plain
CSS commits is significantly diminished.

Having the above limitations in mind, the search provides evidence to conclude that despite the
popularity of preprocessors, plain CSS is still handcrafted on GitHub in the beginning of 2015.



\chapter{Discovering Existing CSS Code Conventions}
\label{sec:discovering}

\section{Research Method}

The primary organization responsible for the specification of CSS has not published an official CSS
style guide. As a result, the CSS community has produced a pool of coding conventions, best
practices, guidelines and recommendations.

To discover existing CSS code conventions, two searches with the keywords \texttt{CSS code
conventions} were made using the search engines \url{http://duckduckgo.com} and
\url{http://google.com}. The first 100 results of each search were analyzed. From each result
only conventions that refer to plain CSS are taken into account and conventions related to
preprocessors are ignored. In case the result contains links to other style guides, these
references are considered as results and analyzed separately.

While searching for conventions, a number of issues were discovered. First, some of the
conventions do not provide sufficient information to be applied in practice. Such an example is the
convention \textit{Do not use CSS hacks - try a different approach first} when the style guide does
not define the meaning of CSS hacks. Such overgeneralized conventions were omitted from the results.

Another part of the discovered conventions introduce a discrepancy between their description in
natural language and the provided code example. An instance of such contradiction is when the
convention \textit{Nothing but declarations should be indented} is followed by a code snippet
illustrating that rules in media queries should also be indented. In such cases the convention is
interpreted as described by the code example.

When conventions are not supported with code examples, their description could remain open for
interpretation. For example, \textit{Rules with more than 4 selectors are not allowed} could be seen
as forbidding multi-selectors with more than four selectors, or disallowing selectors with more than
four simple-selectors. All possible interpretations of ambiguous conventions were registered as
separate conventions.

There are conventions that are not explicitly stated, but could only be inferred by the other rules.
For example, the convention \textit{You can put long values on multiple lines} states that a long
value is allowed to appear on several lines. However, this rule implies the presence of another
convention, that is not explicitly stated. It implies that short values should appear on one line
and only lengthy values are allowed to appear on multiple. Such implicit conventions were registered
as explicit conventions.

\section{Results}

As a result of the searches, a corpus of 165 unique CSS code conventions was accumulated. Sources
of these conventions include CSS professionals, open-source communities and companies, such as \href{https://google-styleguide.googlecode.com/svn/trunk/htmlcssguide.xml#Protocol}{Google}, \href{http://primercss.io/guidelines/#css}{GitHub}, \href{https://make.wordpress.org/core/handbook/best-practices/coding-standards/css/}{Wordpress}, \href{https://www.drupal.org/node/1887862}{Drupal}. The following list presents a selected set of the discovered conventions:

\begin{itemize} 
\item Avoid qualifying ID and class names with type selectors. 
\item Use RGBA only when opacity is needed.
\item Use short hexadecimal values. 
\item CSS files must not include any @charset statements.
\item Disallow @import. 
\item When possible, use em instead of pix. 
\item Avoid using z-indexes when possible. 
\item Require compatible vendor prefixes. 
\item Do not use !important. 
\item Do not use id selectors. 
\item Id and class names should be lowercase.
\item All values except the contents of strings should be lowercase.  
\item HTML tags should be lowercase.
\item Put a ; at the end of declarations. 
\item Do not put quotes in URL declarations.
\item Use short hex values.
\item Use the shorthand margin property.
\item Do not use units after 0 values.
\item Use a leading zero for decimal values.
\item Use single quotes in charsets.
\item Use single quotes in attribute selectors.
\item Put one space between the colon and the value of a declaration.
\item Put one space between the last selector and the block.
\item One selector per line.
\item A rule must not contain height and border, border-top, border-bottom, padding, padding-top, or padding-bottom.
\item A rule must not contain width and border, border-left, border-right, padding, padding-left, or padding-right.
\item Warning if a property is included in a rule twice and contains the same value.
\item Forbid empty rules.
\item A vendor-prefixed property must be followed by a standard property.
\item Require fallback color. A color property with a RGBA(), HSL(), or HSLA() color without a preceding color property that has an older color format.
\item A rule that has display: inline-block should not use float.
\item A rule that has display: block should not use vertical-align.
\item Use 4 spaces for indentation, no tabs.
\item No trailing spaces.
\end{itemize}

The full list of all discovered conventions along with their sources and explanation of their meaning is available in \href{https://github.com/boryanagoncharenko/CssCoco/blob/master/analysis.md}{CssCoco GitHub repository}.


% !TEX root = thesis.tex
\chapter{Expressing CSS Code Conventions}
\label{sec:expressing}

This chapter introduces CssCoco -- a domain--specific language capable of expressing CSS conventions. First, the convention corpus is analyzed and the meaning of conventions is made explicit. Second, the abstract and concrete syntax of CssCoco are defined. Third, an implementation of the language interpreter is presented. Finally, the solution is validated using ontological analysis.

\section{Analysis of conventions corpus}

Code conventions is an umbrella term that comprises rules for whitespacing, comments, indentation,
naming, syntax, code patterns, programming style, file organization etc. To gain an overview of the
type of conventions used in the CSS domain, all conventions in the corpus are organized in groups
depending on the type of constraints they impose. The following three categories were defined
(sublists provide examples of conventions that fall in each category):

\begin{description}
  \item[Layout] category contains rules that constrain the overall layout of the code. It includes conventions related to whitespace, indentation and comments. Examples include:
  \begin{itemize}
  \item Use four tabs for indentation.
  \item Put one blank line between rulesets.
  \item Disallow spaces at the end of the line.
  \end{itemize}
  \item[Syntax Preference] category comprises conventions that express preference of a particular syntax. Note that rules in this category do not aim at ensuring CSS validity, but choose between syntactic alternatives. For example, both single and double quote strings are valid in CSS and a convention may narrow down the choice of the developer to single quotes. Examples include:
  \begin{itemize}
  \item Use lowercase for id and class names.
  \item Require a semicolon at the end of the last declaration.
  \item Use strings with single quotes.
  \end{itemize}
  \item[Programming Style] category consists of conventions that put constraints on how CSS constructs are used to achieve a certain goal. They specify preferred code patterns or anti-patterns. Conventions in this group are used mainly to improve maintenance and performance, or to avoid issues in a particular implementation. Examples are:
  \begin{itemize}
  \item Do not use the universal selector.
  \item Avoid using !important.
  \item A vendor-prefixed property must be followed by a standard property.
  \end{itemize}
\end{description}

Conventions in each of the groups were analyzed and their violations were made
explicit. While the violations of most of the conventions are obvious, some of
them require knowledge about the grammar of CSS. For example, conventions such
as \textit{Avoid id selectors} directly describe their violations --- id
selectors. That said, the convention \textit{Use single quotes in URLs} has
two violations that are valid CSS code: URLs with single quotes and URL
without quotes.

After the violations of each convention were made explicit, the specific actions needed to detect
these violations were determined. Currently, the detection of violations is performed by developers
manually or with the partial help of tools. To perform such checks, developers need to understand
different concepts, e.g. the concept of a rule, HTML element, IDs, etc. and perform certain actions,
such as find a structure, evaluate a constraint etc. The analysis tries to grasp the specific
concepts and actions used to find violations. To illustrate the process, the analysis of one convention is included. Analysis of all conventions in the corpus is available at the CssCoco GitHub repository
\footnote{\url{https://github.com/boryanagoncharenko/CssCoco/blob/master/analysis.md}}.

\bigskip
\begin{boxedminipage}{.95\textwidth}
\begin{description}
\item\textbf{Convention:} Disallow empty rules.
\item\textbf{Author:} \href{https://github.com/CSSLint/csslint/wiki/Disallow-empty-rules}{CSS lint}
\item\textbf{Violations:} Presence of rulesets that do not contain declarations. In case at least one declaration is present, the ruleset does not violate the convention. Examples include:

\begin{lstlisting}[style=mono,language=Java]
.myclass { }                 /* violation */
.myclass { /* Comment */ }   /* violation */
.myclass { color: green; }   /* not violation */
\end{lstlisting}

\item\textbf{Actions:} Recognize rulesets and declarations. Determine whether a ruleset does not contain any declarations.
\end{description}
\end{boxedminipage}
\bigskip

The convention aims at getting rid of one type of refactoring leftovers ---
rulesets without declarations. Removing empty rulesets reduces the total size of CSS that
needs to be processed by the browser. One possible approach for discovering violations of the
convention at hand is to search the stylesheet for rulesets and then check whether each ruleset
contains a declaration. To perform this search successfully, developers need to understand the
concept of a ruleset and a declaration, i.e. they need to be able to recognize these two CSS
structures. Further, developers need to determine relations between structures, particularly,
whether a ruleset contains a declaration.

After all conventions were analyzed, the specific actions and concepts were
used to extract requirements and draw conclusions about the needed
functionality. First, every convention can be represented as a combination of
constraints, regardless of the way it is expressed. There are two major
constructs used to convey conventions in natural language: forbid and require.
Conventions that forbid describe directly their violations. For example, the
convention \textit{Disallow @import} specifies that import statements are
violations. Conventions that use the latter construct describe a pattern and
once the pattern is found, a constraint is evaluated. In case the constraint
is not met, a violation is discovered. For example, the convention
\textit{Class names should be lowercase} requires finding class nodes and then
checking whether they are lowercase.

As the three categories of conventions imply, conventions can reference nodes
from the abstract syntax tree, concrete syntax tree and parse tree~\cite{zaytsev2014parsing} of
CSS. For example, the convention \textit{A rule must
not contain width and padding declarations} accesses concepts that are present
in the abstract syntax tree. Similarly, the rule \textit{Put a semicolon at
the end of the last declaration} refers to nodes that are omitted by the
abstract syntax tree and are present in the concrete syntax tree. All
whitespacing and indentation conventions target nodes that are relevant only
in the parse tree of CSS.

Usually, the patterns described in conventions target one of all described
nodes. For example, when the pattern refers to rulesets in media queries, the
target node is the ruleset and the media query is part of the context.
Similarly, when a pattern describes a ruleset that contains a float
declaration, the target node is the ruleset and the float declaration is only
a context constraint. Certain conventions can have more than one target nodes,
e.g. when two declarations is a ruleset need to be compared.

Conventions refer to nodes using their type or function in the CSS program.
For example, in the snippet \texttt{[class=``test'']} the node test can be
selected 1) because it is of type string and 2) because it is an attribute
value. Similarly, a node with value \texttt{\#ffffff} may be selected because
it is a hexadecimal value or because it represents a color.

Conventions may use CSS-specific knowledge. For example, the rule \textit{Use
lowercase for properties; vendor-prefixed properties are exception} requires
differentiating between standard and vendor-specific properties. While in this
case the two types of properties can be easily distinguished, some conventions
require information that cannot be obtained using the CSS code. Consider the
convention \textit{Order vendor-prefixed values by their version; newer
versions of vendor values should appear after old ones}. To detect violations
for this convention the release dates of the properties need to be available
for comparison.

Conventions rarely target a single node. Typically, they refer to a number of
nodes organized in a pattern. For example, the convention \textit{A ruleset
must not contain display and float declarations} requires searching for two
specific declaration that appear under the same parent node. The nodes in the
pattern do not have to be immediate relatives. In fact, they can be scattered
across the tree. For example, the rule \textit{Do not use more than 5
@font-face declarations} requires searching for specific nodes that appear
anywhere in the tree.

\section{CssCoco DSL}

\subsection{Syntax overview}

To express the conventions in the corpus, the domain-specific language CssCoco
is proposed. It is a declarative language that has two main constructs:
conventions and contexts. Conventions express the specific rules that has to
be enforced on the code and contexts describe the CSS nodes that need to be
ignored while searching for violations.

The language constructs that define conventions try to resemble the way
conventions are expressed in natural text. There is a construct that describes
directly what is disallowed. For example, the convention \textit{Do not use
import statements} is expressed in the following way:

\begin{sourcecode}
\begin{lstlisting}[style=mono,language=Java]
forbid import
message 'Do not use import statements.'
\end{lstlisting}
\end{sourcecode}

The keyword \textit{forbid} is followed by a description of the node that is
disallowed. In the current convention, we only need to state its type ---
import. Each convention requires a message clause. The string after the
\textit{message} keyword will be displayed to the user when a violation is
found.

As domain analysis indicates, conventions are often expressed as a pattern that, if found, needs to
meet given constraints. In CssCoco syntax such conventions are defined using the \texttt{find ...
require ...} construct. For example, the convention \textit{All class names should be lowercase} is
described as follows:

\begin{sourcecode}
\begin{lstlisting}[style=mono,language=Java]
find c=class
require c.name match lowercase
message 'All class names should be lowercase'
\end{lstlisting}
\end{sourcecode}

The find clause in the above rule specifies the pattern that needs to be found and the require
clause states the constraint that should be applied to the discovered nodes. Note that to refer to a
matched node in the require clause, the node should be assigned as identifier, \texttt{c} in the
above example.

Conventions can put more constraints on a node description in the pattern. For example, the
convention \textit{Use a leading zero for decimal values} requires finding all nodes of type number
that have a numeric value in the interval [-1, 1]. Such constraints are expressed as curly brackets
immediately after the type of the node.

\begin{sourcecode}
\begin{lstlisting}[style=mono,language=Java]
find n=number{num-value < 1 and num-value > -1}
require n.string match '^0.*'
message 'Use a leading zero for decimal values'
\end{lstlisting}
\end{sourcecode}

Conventions can describe patterns that consist of more than one node. For example, the rule
\textit{Use single quotes in charsets} can be expressed in the following way:

\begin{sourcecode}
\begin{lstlisting}[style=mono,language=Java]
find s=string in charset
require s.has-single-quotes
message 'Use single quotes in charsets'
\end{lstlisting}
\end{sourcecode}

The pattern contains the description of two nodes: first, a node that is of type string and a node
of type charset. The \texttt{in} keyword description of the pattern indicates that the string node
is nested in the charset node. In this way, the pattern will match only the strings that appear in a
charset.

In CssCoco, conventions are grouped in contexts that specify what nodes should be ignored when
searching for patterns. For example, often when rules refer to newlines they completely disregard
indentation. The convention \textit{Every declaration must be on a new line} requires a newline to
be present immediately before the declaration. However, when declarations are indented their
immediate previous sibling is an indentation node. To handle such cases, the language uses contexts
that explicitly describe the ignored nodes.

\begin{sourcecode}
\begin{lstlisting}[style=mono,language=Java]
Whitespace
ignore indent 
{
    find d=declaration
    require newline before d
    message 'Put every declaration on a new line'
}
\end{lstlisting}
\end{sourcecode}

Contexts have a user-defined name and an optional ignore clause. The \texttt{ignore} keyword is
followed by a description of the nodes that need to disregarded.

\subsection{Abstract syntax}

This section describes the abstract syntax of the designed domain-specific language. An overview of
the abstract syntax is presented in \autoref{fig:astgeneral}, followed by detailed views of each
of the subclasses.

\begin{figure}[h]
  \centering
  \caption{Abstract Syntax Overview}
  \label{fig:astgeneral}
  \includegraphics[width=\textwidth]{general}
\end{figure}


\begin{description}

\item\textbf{ConventionSet} represents a style guide. It comprises a number of conventions that form coherent guidelines. Attribute \texttt{contexts} is a list of Contexts that contain conventions.


\item\textbf{Context} represents a group of conventions that belong to the same semantic group (e.g. whitespacing, syntax preference, programming style). Attribute \texttt{conventions} is a list of Contexts that contain conventions. Attribute \texttt{ignored\_patterns} is a list of Patterns that are ignored while searching for the target pattern. For example, while searching for violations of semantic conventions, the whitespacing and indentation nodes are ignored.

\item\textbf{Convention} represents a rule that enforces specific constraints. Attribute \texttt{pattern} is the pattern that the convention targets. Attribute \texttt{description} is the description of the convention in natural text. This description is displayed to the user when a violation of the convention is discovered.

\item\textbf{PatternDescriptor} represents a description of a node or a combination of related nodes that given convention constraints. Attribute \texttt{root} is the top node described in the pattern. Attribute \texttt{nodes} is a collection of all nodes described by the pattern. Attribute \texttt{relations} is a collection of relationships between the nodes used in the pattern.

\item\textbf{NodeDescriptor} is an abstract class that contains a description of a Css Node. Attribute
\texttt{constraint} is an expression that designates constraints applied to the node.
Attribute \texttt{identifier} is a given string that can be used as a reference to the matched node.

\item\textbf{Node} represents a description of a node used in a PatternDescriptor. Attribute
\texttt{constraint} is an expression that designates constraints applied to the node.
Attribute \texttt{identifier} is a given string that can be used as a reference to the matched node.

\item\textbf{WhitespaceNode} represents a description of a whitespace node that references space,
newline, indentation symbols. Attribute \texttt{constraint} is an expression that designates
constraints applied to the node. Attribute \texttt{repeater} is an optional constraint that specific
the number of times a whitespace node can appear consecutively. Repeaters are useful to express
conventions that do not specify exact quantities of whitespace symbols. For example, the convention
``put at least one blank line between rules'' sets a lower limit of the number of blank lines, but
not an upper limit.

\item\textbf{NodeRelation} represents a relation between two Nodes. Attribute \texttt{target\_node}
designates a description of the Node targeted by the relation.

\end{description}


\begin{figure}[h]
  \centering
  \caption{Abstract Syntax Expressions}
  \label{fig:expressions}
  \includegraphics[width=\textwidth]{expr}
\end{figure}

A detailed view of the expressions in the abstract syntax of CssCoco is presented in \autoref{fig:expressions}. Following is a description of the subclasses of Expression:

\begin{description}

\item\textbf{LiteralExpr} represents an expression containing a literal value. Attribute \texttt{value} is the value of the literal expression.

\item\textbf{VariableExpr} represents a reference to a matched node. Attribute \texttt{name} is the identifier used to reference the node.

\item\textbf{UnaryExpr} represents expressions with a single operand. Attribute \texttt{operand} is operand of the expression.

\item\textbf{NotExpr} represents logical negation expression.

\item\textbf{UnaryMinusExpr} represents unary minus expression.

\item\textbf{BinaryExpr} represents expressions with a two operands.

Attributes \texttt{left} and \texttt{right} represent the first and second operands, respectively.

\item\textbf{OrExpr} represents logical disjunction expression.

\item\textbf{AndExpr} represents logical conjunction expression.

\item\textbf{ComparisonExpr} represents expression that compares two operands.

\item\textbf{IsExpr} represents expression that checks whether the first operand is of the given type, specified by the second operand.

\item\textbf{InExpr} represents expression that checks whether the first operand is present in a list of values, specified by the second operand.

\item\textbf{MatchExpr} represents expression that checks whether the first operand matches a regular expression, specified by the second operand.

\item\textbf{CallExpr} represents expression that invokes a API property or method of the operand. Attribute \texttt{operand} is the operand of the expression. Attribute \texttt{value} is the name of the API property or method that is invoked.

\item\textbf{NodeQueryExpr} represents expression that queries node context. Attribute \texttt{operand} is the node used as a reference point for the query.

\end{description}

\begin{figure}[h]
  \centering
  \caption{Abstract Syntax Literal Expressions}
  \label{fig:literal}
  \includegraphics[width=0.9\textwidth]{literal}
\end{figure}

\autoref{fig:literal} presents a detailed view of the literal expressions used in the abstract syntax of CssCoco. Following is a listing of the classes. 

\begin{description}

\item\textbf{IntegerExpr} represents expression containing a integer value.

\item\textbf{DecimalExpr} represents expression containing a decimal value.

\item\textbf{StringExpr} represents expression containing a string value.

\item\textbf{BooleanExpr} represents expression containing a boolean value.

\item\textbf{ListExpr} represents expression containing a list value. The elements of the list are of type LiteralExpr.

\item\textbf{NodeTypeExpr} represents expression containing a string value that describes node type.

\end{description}

\begin{figure}[h]
  \centering
  \caption{Abstract Syntax Literal Expressions}
  \label{fig:call}
  \includegraphics[width=0.2\textwidth]{call}
\end{figure}

\autoref{fig:call} presents a detailed view of the call expressions used in the abstract syntax of CssCoco. Following is a listing of the classes. 

\begin{description}

\item\textbf{PropertyExpr} represents an expression that returns the value of a property of the operand node. Attribute \texttt{operand} represents the node targeted by the expression. Attribute \texttt{value} holds the name of the property that is accessed.

\item\textbf{MethodExpr} represents an expression that returns invokes a method of the operand node. Attribute \texttt{argument} represents argument passed to the invoked method.

\end{description}

\begin{figure}[h]
  \centering
  \caption{Abstract Syntax Node Query Expressions}
  \label{fig:nodequery}
  \includegraphics[width=0.9\textwidth]{nodequery}
\end{figure}

\autoref{fig:nodequery} presents an overview of the Node Query Expressions. The following listing describes the subclasses in details:

\begin{description}

\item\textbf{NextSiblingExpr} represents expression that returns the following sibling of the operand node.

\item\textbf{PreviousSiblingExpr} represents expression that returns the previous sibling of the operand node.

\item\textbf{NodeQueryWithArgExpr} represents expression that queries node context and uses additional constraints for the query. Attribute \texttt{argument} represents the additional constraints used by the query.

\item\textbf{ContainsExpr} represents an expression that checks whether the operand node contains a node that matches given constraints.

\item\textbf{ContainsAllExpr} represents an expression that checks whether the operand node contains nodes that match given constraints.

\item\textbf{CountExpr} represents an expression that counts the number of ancestor nodes of the operand that match a given constraint.

\item\textbf{BeforeExpr} represents an expression that checks whether a given whitespace variation appears before the operand node.

\item\textbf{AfterExpr} represents an expression that checks whether a given whitespace variation appears after the operand node.

\item\textbf{BetweenExpr} represents an expression that checks whether a given whitespace variation appears between the two operand nodes.

\end{description}

\subsection{Concrete syntax}

This section contains the concrete syntax of the designed DSL. Below are presented the grammar rules accompanied by the mapping to the abstract syntax of the language.


\begin{description}

\item\textbf{stylesheet} represents a style guide.

Abstract Syntax Mapping: ast.ConventionSet.

\begin{snippet}
\begin{verbatim}
stylesheet : context* ;
\end{verbatim}
\end{snippet}

\item\textbf{context} represents a group of logically related conventions. A single style guide can comprise a number of conventions that enforce various constraints, e.g. whitespacing, syntax preference, program style. Contexts group conventions that ignore the same nodes while searching for their violations.  

Abstract Syntax Mapping: ast.Context.

\begin{snippet}
\begin{verbatim}
context : Identifier ignore_clause? '{' convention* '}' ;
ignore_clause : 'ignore' (node_descriptor)+ (',' (node_descriptor)+)* ;
\end{verbatim}
\end{snippet}

\item\textbf{convention} represents a single rule in the style guide. Conventions are typically expressed by directly stating what is disallowed or describing a condition that if met, requires additional constraints. The former way of expressing conventions are represented by the \texttt{forbid} conventions. The latter approach uses the structure \texttt{find ... require ...}. To break down complex disallowing conventions, the structure \texttt{find ... forbid ...} has been introduced. This aims at improving readability of conventions. Additionally, the find conventions have a where clause which applies constraints for matching nodes. It is used to expression matching constraints that span over multiple nodes and therefore cannot be present in any of the node descriptors.

Abstract Syntax Mapping: ast.Convention.

\begin{snippet}
\begin{verbatim}
convention : 'forbid' pattern 'message' String
           | 'find' pattern ('where' logic_expr)? ('require'|'forbid') logic_expr 'message' String
           ;
\end{verbatim}
\end{snippet}

\item\textbf{pattern} represents a pattern of nodes and their relations. For example, it can describe a horizontal sequence of sibling nodes, a vertical pattern of nested nodes, or pairs of elements with a common parent. 

Abstract Syntax Mapping: ast.PatternDescriptor.

\begin{snippet}
\begin{verbatim}
pattern : node_declaration (('in'|'next-to') node_declaration)*
        | fork ('in' node_declaration)*
        ;
fork : '(' node_declaration (',' node_declaration)+ ')' ;
node_declaration : (Identifier '=')? semantic_node ;
\end{verbatim}
\end{snippet}

\item\textbf{node\_descriptor} represents a description of a node. It describes the type of the node and its additional constraints. 

Abstract Syntax Mapping: ast.NodeDescriptor.

\begin{snippet}
\begin{verbatim}
node_descriptor : 'unique'? node_type ('{' (logic_expr|repeater) '}')? ;
repeater : Integer ',' Integer? | (',')? Integer ;
\end{verbatim}
\end{snippet}


\item\textbf{logic\_expr} represents expressions that perform logic operations and glue arithmetic and type expressions. 

Abstract Syntax Mapping: ast.NotExpr, ast.AndExpr, ast.OrExpr and all arithmetic\_expression and type\_expression mappings.

\begin{snippet}
\begin{verbatim}
logic_expr : '(' logic_expr ')'
           | 'not' logic_expr
           | logic_expr 'and' logic_expr
           | logic_expr 'or' logic_expr
           | type_expr
           | arithmetic_expr
           ;
\end{verbatim}
\end{snippet}

\item\textbf{type\_expr} represents expressions that ensure node type and perform node queries on nodes. They are located in a separate parser rule because they interpret Identifiers as node type expressions instead of a API calls. 

Abstract Syntax Mapping: ast.IsExpr, ast.BeforeExpr, ast.AfterExpr, ast.BetweenExpr.

\begin{snippet}
\begin{verbatim}
type_expr : arithmetic_expr operator='is' Identifier
          | node_descriptor+ ('before' | 'after') type_operand
          | node_descriptor+ 'between' type_operand 'and' type_operand
          ;
type_operand : Identifier | semantic_node ;
\end{verbatim}
\end{snippet}

\item\textbf{arithmetic\_expr} represents arithmetic, comparison, set membership and regex expressions. These are located in a separate parser rule because they interpret identifiers as API calls instead of node type expressions.

Abstract Syntax Mapping: ast.UnaryMinus, ast.UnaryPlus, ast.LessThan, ast.LessThanOrEq, ast.GreaterThan, ast.GreaterThanOrEq, ast.Equal, ast.NotEqual, ast.InExpr, ast.MatchExpr, ast.LiteralExpr.

\begin{snippet}
\begin{verbatim}
arithmetic_expr : ('-'|'+') arithmetic_expr
                | arithmetic_expr ('<'|'>'|'<='|'>='|'=='|'!=') arithmetic_expr
                | arithmetic_expr ('in'|'not in'|'match'|'not match') arithmetic_expr
                | call_expression
                | element
                ;
element : Boolean | Decimal | Integer | String | list_ ;
\end{verbatim}
\end{snippet}

\item\textbf{call\_expression} represents an API call expression and also node query expression.

Abstract Syntax Mapping: ast.CallExpr and ast.NodeQueryExpr.

\begin{snippet}
\begin{verbatim}
call_expression : call_expression '.' call_expression
                | Identifier ('(' (element | semantic_node ) ')')? 
                ;
\end{verbatim}
\end{snippet}

\item\textbf{Boolean:} represents Boolean literal expression. 

Abstract Syntax Mapping: ast.BooleanExpr.

\begin{snippet}
\begin{verbatim}
Boolean : 'true' | 'True' | 'false' | 'False' ;
\end{verbatim}
\end{snippet}

\item\textbf{String:} represents String literal expression.

Abstract Syntax Mapping: ast.StringExpr.

\begin{snippet}
\begin{verbatim}
String : "'" (EscapeSequence | ~['])*? "'" ;
EscapeSequence : "\\" "'" ;
\end{verbatim}
\end{snippet}

\item\textbf{Integer:} represents Integer literal expression. 

Abstract Syntax Mapping: ast.IntegerExpr.

\begin{snippet}
\begin{verbatim}
Integer : (ZeroDigit | NonZeroDigit Digit*) ;
Digit : ZeroDigit | NonZeroDigit ;
NonZeroDigit : [1-9] ;
ZeroDigit : [0] ;
\end{verbatim}
\end{snippet}

\item\textbf{Decimal:} represents Decimal literal expression. 

Abstract Syntax Mapping: ast.DecimalExpr.

\begin{snippet}
\begin{verbatim}
Decimal : ( NonZeroDigit Digit* | ZeroDigit? ) '.' Digit+ ;
Digit : ZeroDigit | NonZeroDigit ;
NonZeroDigit : [1-9] ;
ZeroDigit : [0] ;
\end{verbatim}
\end{snippet}

\item\textbf{list} and \textbf{list\_element} represent the List literal expression. 

Abstract Syntax Mapping: ast.ListExpr.

\begin{snippet}
\begin{verbatim}
list_ : '[' list_element (',' list_element)* ']' ;
list_element : Integer | Decimal | String | semantic_node ;
Letter : [a-zA-Z] ;
Identifier : (Letter)(Letter|Digit|'_'|'-')* ;
\end{verbatim}
\end{snippet}

\item\textbf{type\_expression} represents the NodeType literal expression.

Abstract Syntax Mapping: ast.NodeType.

\begin{snippet}
\begin{verbatim}
node_type : '(' node_type ')'
          | 'not' node_type
          | node_type 'and' node_type
          | node_type 'or' node_type
          | Identifier
          ;
\end{verbatim}
\end{snippet}

\end{description}

\subsection{Proof of Concept}

To study the feasibility of the designed language, a proof of concept was
developed. The implemented solution consists of two parts: a standalone
Python package and a plug-in for Sublime Text editor.

The first part of the designed proof of concept comprises the CssCoco
interpreter. The implementation is done in Python and currently contains 13
000 lines of code. The source code is available at CssCoco GitHub repository
\footnote{\url{https://github.com/boryanagoncharenko/CssCoco}}. The solution was also
added to the Python Package Index (Pypi) repository
\footnote{\url{https://pypi.python.org/pypi?:action=display&name=csscoco}}
and for less than month it has accumulated over 4500 downloads. Once
installed, the package offers a \texttt{csscoco} command that takes as
parameters a css and a coco file and returns a list of the discovered
violations.

The current implementation of the CssCoco interpreter is available only for
Python 3.4. Additionally, the proof of concept requires nodejs. This
dependency is added because, the only existing CSS parser that produces CSS
parse trees with the required level of details is implemented in nodejs.

The second part of the proof of concept brings the functionality implemented
in the Python package to Sublime Text editor. The plug-in uses the
\texttt{csscoco} command to find violations in CSS files that are being edited
in the text editor. Currently, the plug-in is implemented for Sublime Text 3.
The source code of the solution is publicly available at a separate GitHub repository
\footnote{\url{https://github.com/boryanagoncharenko/Sublime-CssCoco}}.

The plug-in offers a command that finds and visualizes violations. Similarly to
other linter tools, rows that contain violations are marked with a color
border and a gist appears at the side bar. When the cursor is positioned on a
line that contains a violation, the error message is displayed in the status
bar. For example, on \autoref{sublime} the cursor is placed on line 26 and the
status bar indicates that there should be one space between the colon and the
value of the declaration.

\begin{figure}[h]
  \centering
  \caption{
    \label{sublime}
    CssCoco Sublime Text Plug-in}
  \includegraphics[width=0.7\textwidth]{sublime}
\end{figure}

The proof of concept leaves some of the features of the language not implemented. Specifically, is does not include the following:

\begin{itemize}

\item Unique construct. A small portion of the conventions require the unique construct. The feature is not included in the current version and is left for future implementation.

\item Ordering rules. To check for violations of conventions that specify ordering, requires efficient implementation especially for large CSS files. The feature is currently not supported because of time constraints. 

\item Indentation rules. Conventions that refer to indentation specify relative indentation. For example, when a convention states that the contents of a ruleset should be indented, this implies that they should be indented once compared to the beginning of the ruleset. Detecting violations of such conventions is laborious and is left for future implementation.


\end{itemize}

\section{Validation}

The method chosen for validating the designed domain-specific language is
ontological analysis, since it is a widely accepted way for evaluating
software notations
~\cite{opdahl2002ontological,green2000integrated,moody2009physics,parsons1997using,weber1996analytical}.
The particular approach used for conducting ontological analysis consists of
several steps. First, a domain-specific ontology is designed. Second, the
ontology is used as a reference point for the interpretation and
representation mappings. Third, emerged anomalies are analyzed and conclusion
about the quality of the notation is made.

\subsection{Ontology design}

The first stage of validation requires designing a domain-specific ontology.
The specific domain of the developed ontology is limited to detecting
violations of CSS code conventions. In other words, the designed ontology
tries to capture only the concepts and their relations, that exist when an
agent searches a CSS program for violations of given set of code conventions.

The designed domain-specific ontology is based on the BWW top-level
ontology~\cite{wand1990ontological}, i.e. it uses the high-level categories of
the BWW ontology to describe the objects, concepts and entities in the
specific domain. The rationale behind the decision to use BWW ontology is that
it has been the leading ontology used for ontological
analysis~\cite{moody2009physics}. The main ontological constructs used in the
BWW ontology are listed in section 2.1.

The designed ontology is presented using several approaches. As recommended by
Wand and Weber, the ontology is described using a dictionary comprising
definitions of entities in natural text and, second, using Backus-Naur Form
(BNF) notation~\cite{wand1995deep,rosemann2002developing}. Additionally, a
system diagram is included to provide a better view of the couplings between
the different entities. The ontology is intentionally not presented using
Unified Modeling Language or Entity-Relationship diagrams. These modeling
languages are subjects of ontological analysis themselves and therefore are
not suitable for expressing an ontology.

Following is a list with the main concepts discovered in the domain along with
their descriptions. The used BWW concepts are written in \textit{italics} and
the domain-specific concepts are written in \textbf{bold}.

\begin{description}

\item\textit{Class} \textbf{Style Guide} describes the coding practices
adopted  in the context of a single project, organization, community or
language. An individual Style Guide is a \textit{composite thing} built of
Conventions and their relations. Conventions in a Style Guide are interpreted
together to form a coherent set of guidelines.

\textit{Property} \textbf{Conventions} refers to the conventions contained in the Style Guide.


\item\textit{Class} \textbf{Convention} is a specific rule that imposes constraints on the CSS program. It is the building block of Style Guides. An individual Convention is a \textit{composite thing} that contains a Context.

\textit{Intrinsic Property} \textbf{Description} contains the reasoning behind the Convention.

\textit{Hereditary Property} \textbf{Ignored Constructs} denotes the description of constructs that should be ignored while searching for the Convention's Context. It is inherited by the Context thing that builds a Convention.


\item\textit{Class} \textbf{Context} is a description of a Pattern that the Convention forbids. An individual Context is a \textit{composite thing} that comprises a number of logically related Constraints. When a Pattern in the current Stylesheet fulfills all Constraints in the Context, a Violation is discovered.

\textit{Property} \textbf{Ignored Constructs} are descriptions of Patterns that need to be disregarded while searching for the current Context. In fact, the property denotes a collection of Contexts.


\item\textit{Class} \textbf{Constraint} is a specific restriction that needs to be fulfilled. They
are used in a \textbf{Context} to build a description of a \textbf{Pattern}. Constraints are
individual requirements that are imposed on subjects. Based on the value of the requirement, there
are different types of Constraints represented below as \textit{subclasses}.

\textit{Property} \textbf{Subject} indicates the thing is being constrained.

\textit{Property} \textbf{Requirement} denotes the particular limitation applied to the Subject.


\item\textit{Subclass} \textbf{Existence Constraint} is a type of Constraint that requires existence of the subject. 

\textit{Property} \textbf{Subject} indicates the thing is being constrained.

\textit{Property} \textbf{Requirement} denotes the particular limitation applied to the Subject. Specifically, that the Subject must exist.


\item\textit{Subclass} \textbf{Comparison Constraint} is a type of Constraint that compares the subject to another value.

\textit{Property} \textbf{Subject} indicates the thing is being constrained.

\textit{Property} \textbf{Requirement} denotes the particular limitation applied to the Subject. Specifically, that the Subject must be related to the Value in a given way.

\textit{Property} \textbf{Value} denotes the value that is used for the comparison.


\item\textit{Subclass} \textbf{Type Constraint} is a type of Constraint that checks whether the subject is of a given type.

\textit{Property} \textbf{Subject} indicates the thing is being constrained.

\textit{Property} \textbf{Requirement} denotes the particular limitation applied to the Subject. Specifically, that the Subject must be of the given type.

\textit{Property} \textbf{Value} denotes the type that the subject should meet to satisfy the constraint.


\item\textit{Subclass} \textbf{Textual Form Constraint} is a type of Constraint that imposes restrictions on the textual representation of the subject.

\textit{Property} \textbf{Subject} indicates the thing is being constrained.

\textit{Property} \textbf{Requirement} denotes the particular limitation applied to the Subject. Specifically, that the Subject must be equal to the given Value.

\textit{Property} \textbf{Value} denotes the textual form that the Subject should meet for the constraint to be satisfied.


\item\textit{Subclass} \textbf{Set Membership Constraint} is a type of Constraint that requires the subject to be a member of a set.

\textit{Property} \textbf{Subject} indicates the thing is being constrained.

\textit{Property} \textbf{Requirement} denotes the particular limitation applied to the Subject. Specifically, that the Subject must be a member of the Value.

\textit{Property} \textbf{Value} denotes the set that the subject should be present at for the constraint to be satisfied.


\item\textit{Class} \textbf{Literal Value} is a thing that represents a constant value. It includes numbers, strings, boolean values etc.

\textit{Property} \textbf{Value} denotes the specific value possessed by the literal.


\item\textit{Class} \textbf{Violation Log} is the final product of a violations search. An individual Violation Log is a composite thing that contains Violations.

\textit{Property} \textbf{Number of Violations} indicates the size of the Violation Log. 


\item\textit{Class} \textbf{Violation} A Violation occurs when a Pattern that matches the Context of a Convention is found.

\textit{Property} \textbf{Description} explains in natural text what causes the Violation. Typically, the Description is extracted from the Convention that the Violation breaks.

\textit{Property} \textbf{Position in Stylesheet} indicates the location of the Pattern that violates the Convention in the Stylesheet. 


\item\textit{Class} \textbf{Stylesheet} is the CSS code that needs to be checked for compliance with the Style Guide. An instance of Stylesheet is a composite thing that comprises a number of Constructs.

\textit{Property} \textbf{Checked} indicates whether a Stylesheet has been checked for compliance to a given Style Guide. 


\item\textit{Class} \textbf{Construct} is a part of the Stylesheet. It can refer to nodes in the CSS abstract syntax tree, concrete syntax tree and parse tree. Examples include whitespacing, indentation, comments, colons, delimiters, rulesets, declarations, etc.

\textit{Property} \textbf{Property} encapsulates properties of nodes specific to the CSS domain. For example, the type and the string representation of the node are its properties. Similarly, specific CSS Nodes can expose properties that are tightly coupled to the CSS domain, such as release date or vendor name of a CSS property.


\item\textit{Class} \textbf{Pattern} is a particular part of the CSS program that matches the description of a Context. An instance of a Pattern is a composite thing built from one or many Constructs and Relations between them.

\textit{Property} \textbf{Constructs} denotes the constructs that are contained in the Pattern.


\item\textit{Event} \textbf{Search for Violations in Stylesheet} occurs when the developer completes the search for violations in a Stylesheet, a Violation Log is created and the state. When the search is completed, the Stylesheet is considered checked for compliance to the Style Guide.

\textit{New State} \textbf{Violation Log} \{ Violations = value \}

\textit{New State} \textbf{Stylesheet} \{ Checked = True \}


\item\textit{Event} \textbf{Context (Convention) Discovered} occurs when the Context of a convention is discovered, a Violation is recorded in the Violation Log. The state of the Violation contain its description and position in Stylesheet.

\textit{New State} \textbf{Violation} \{ Description = value, Position in Stylesheet = value \}


\item\textit{Event} \textbf{Stylesheet modified} occurs when the Constructs in the Stylesheet are modified. The state of the Stylesheet is changed to unchecked for compliance.

\textit{New State} \textbf{Stylesheet} \{ Checked = False \}


\item\textit{Event} \textbf{Style Guide modified} occurs when any of the parts of a Style Guide are modified. This event changes the state of the Stylesheet to unchecked for compliance.

\textit{New State} \textbf{Stylesheet} \{ Checked = False \}

\end{description}


Most of the definitions in the ontology refer to simple concepts that appear in the code conventions
domain. For example the concept of a Style Guide refers to a collection of coherent conventions. A
Style Guide on its own does not have any emergent or intrinsic properties and, thus, it is defined
through the conventions it comprises. Note that in reality a Style Guide may contain a number of
intrinsic properties, for example it may have an author and contributors. Such
properties, however, are not considered part of the specific domain, and thus lie outside the scope
of the ontology.

A Convention is defined as the building block of a Style Guide. However, the ontological concept of
a Convention is slightly different than what is used as a convention in the domain analysis section.
For example, domain analysis indicates that there are two types of conventions: forbid and require.
The former type of conventions describes directly their violations. The latter type describes a
pattern that, when found, additional requirements must be met and in case they are not met, a
violation is discovered. For example, the violation \textit{Do not use id selectors} tells that id
selectors are violations. The convention \textit{Strings should be with single quotes} states that
strings must have single quotes, and a violation is discovered if the found string does not.
These convention types are not related to the meaning of conventions but to the way they are expressed. A
convention with the same meaning could be expressed using both structures: \textit{Strings should be
with single quotes} and \textit{Forbid strings with double quotes}. Since ontological concepts are
concerned with the meaning of things and have to be independent of the language used to express
them, the ontology does not possess subclasses of Convention.

A Context is a description of things that are disallowed by a Convention, i.e. it states explicitly
what Patterns violate a Convention. In this sense, the meaning of a Convention is always expressed
through the possible violations of that Convention. A Context is built from a number of Constraints.
While a Context aims at describing a whole violation pattern, a Constraint refers to a single
specific requirement. A Constraint always has a subject and a requirement. The subject is the object
that is being constraint and the requirement refers to the specific limitation that is applied to
the subject. Based on the value of the requirement property, there are different types of
Constraints. For example, a Comparison Constraint requires the subject to be equal or greater than a
given value and the Type Constraint require the subject to be of a particular type. The Existence
Constraint requires the subject to exist.

Since Conventions are used to constraint CSS code, subjects are typically Constructs or their
properties. A Construct denotes a concrete part of the Stylesheet, e.g. newline, semicolon,
declaration, ruleset etc. Each Construct exposes a number of properties that are tightly coupled to
its function in the stylesheet. For example, a property construct has properties
that indicate whether it is vendor-specific and its release date.

Patterns are composed of Constructs. The Constructs in the Pattern do not need to form a coherent
valid Stylesheet and they do not have to be adjacent or directly related. In fact, they could be
scattered across the whole Stylesheet. Patterns denote the concrete Constructs that match the
description provided by the Context of a Convention. In this sense, they are the specific instances
of Violations.

Definitions of the ontological concepts in the listing above often state that an instance of a
class is a composite thing that consists of other things. To provide a better understanding of the
way composite things are constructed, the same concepts are also expressed using BNF notation:

\begin{snippet}
\begin{verbatim}
style_guide    ::= convention+
convention     ::= context
context        ::= constraint+
constraint     ::= existence_constraint | comparison_constraint | type_constraint  
                 | text_constraint | set_constraint
violation_log  ::= violation*
stylesheet     ::= construct+
pattern        ::= construct+
\end{verbatim}
\end{snippet}

The grammar above illustrates that a Style Guide needs to contain one or more Conventions.
Similarly, a Context requires at least one Constraint in order to exist. A Violation Log, however,
could exist without any Violations in the cases when a Stylesheet is checked for conformance to a
Style Guide and no violations are discovered. Both Stylesheet and Pattern are defined through 1 or
more Constructs. Note that the concept of Stylesheet does not map to a CSS file, but to the whole
CSS code that needs to be processed, regardless of type of CSS. This is why a Stylesheet requires at
least one Construct in order to exist.

While the above grammar presents the composition of things, it does not illustrate how things
interact with each other. To provide a better understanding of the dynamics between things defined
in the ontology, a graph of the system is presented in \autoref{fig:couplings}.

\begin{figure}[h]
  \centering
  \caption{Graph of the system}
  \label{fig:couplings}
  \includegraphics[width=0.8\textwidth]{couplings}
\end{figure}

According to the theory, a coupling occurs when the existence of a given thing affects the history
of another thing and, in turn, history is defined as the chronological ordered states that a thing
traverses~\cite{wand1990ontological}. Thus, in the domain-specific ontology a coupling exists
between the Style Guide and the Convention things, because the existence of a Convention alters the
state of the Style Guide. Similarly, a Context changes the state of a Convention and a Constraint
affects the state of a Context.

There are also couplings between Construct, Pattern and Stylesheet things. Both Stylesheet and
Pattern are composed of Constructs, and thus affected by their existence. Since a Pattern is a
specific occurrence of a combination of Constructs, it is also coupled to Stylesheet.

A Violation is coupled to a Violation Log, since the presence of a new a Violation alters the state
of the log. Further, a Violation Log contains information about the violations of a particular Style
Guide that occur in a specific Stylesheet. In this sense, a Violation Log is a function of a Style
Guide and a Stylesheet and it is coupled to both things.

There are a number of external events that can change the state of the system. An actor can initiate
search for violations, which affects the state of the Violation Log and the Stylesheet. If during
the search an actor discovers a violation, the state of the Violation is altered. Also, an actor can
modify the Stylesheet and the Style Guide. Thus, couplings exist between the Actor and the
Stylesheet, Violation, Violation Log and Style Guide.

\subsection{Ontological analysis}

The ontological analysis is a bidirectional mapping between the designed domain-specific ontology and the
domain-specific language. It consists of two mappings: representation and interpretation~\cite{moody2009physics}. The former
mapping matches the ontology to the language and the latter --- the language to the ontology.
Typically, ontological analysis is used to compare the abstract syntax of the language constructs to
the concepts of an ontology. However, the designed ontology contains concepts that fall outside the
scope of the abstract syntax. For example, the notion of Violation Log has a corresponding class in
the system, however, that class denotes the product of the violations search and is not part of the
abstract syntax. The particular approach chosen to conduct ontological analysis is to compare the
abstract syntax of the language to the ontology. Additionally, all remaining ontological constructs
are mapped to their representation in the whole system. For example, the ontological concept of a
Violation Log is matched with the Violation Log class in the analysis module.

\subsubsection{Representation and Interpretation Mappings} 

This subsection contains the two mappings between the modeling grammar and domain-specific ontology.
\autoref{tab:representation} presents the representation mapping. The left-hand side of the table
contains the ontological constructs in the order of definition and the right-hand side comprises the
corresponding modeling grammar construct. \autoref{tab:interpretation} presents the interpretation
mapping. The left-hand side contains classes from the abstract syntax, followed by classes that have
a representation in the ontology and the right-hand side contains the corresponding ontological
construct.

\begin{table}[t]
\begin{center}
\begin{longtable}{ | p{20em} | p{20em} | } 
\hline
 &  \\
\textbf{Ontological constructs} & \textbf{Grammar constructs} \\
 &  \\
\hline
Style Guide & Convention Set \\ \hline
Conventions (Style Guide) & Contexts (Convention Set), Conventions (Context) \\ \hline
Convention & Convention \\ \hline
Description (Convention) & Description (Convention) \\ \hline
Context & Pattern Descriptor \\ \hline
Ignored Constructs (Context) &  Context \\ \hline

Constraint & Comparison, Is, In, Match, Node Query, Not, And, Or Expressions \\ \hline
Subject (Constraint) & Operand of Expression \\ \hline
Requirement (Constraint) & Subclasses of Expression \\ \hline
Value (Constraint Subclasses) & Second Operand of Binary Expression \\ \hline
Existence Constraint & Node Descriptor, Node Relation, Node Query Expression \\ \hline
Comparison Constraint & Comparison Expression \\ \hline
Type Constraint & Is Expression \\ \hline
Textual Form Constraint & Match Expression \\ \hline
Set Membership Constraint & In Expression \\ \hline
Literal Value & Literal Expression \\ \hline
Value (Literal Value) & Value (Literal Expression) \\ \hline

Violation Log & Violation Log \\ \hline
Violations (Violation Log) & Violations (Violation Log) \\ \hline
Violation & Violation \\ \hline
Description (Violation) & Description (Violation) \\ \hline
Position in Stylesheet (Violation) & Position (Violation) \\ \hline

Stylesheet & Stylesheet Node \\ \hline
Checked (Stylesheet) & - \\ \hline
Construct & Variable Expression \\ \hline
Property (Construct) & Call Expression \\ \hline
Pattern & Css Pattern \\ \hline
Constructs (Pattern) & Nodes (Css Pattern) \\ \hline
\end{longtable}
\end{center}
\caption{
	\label{tab:representation}
	Representation mapping}
\end{table}

\begin{table}[t]
\begin{center}
\begin{longtable}{ | p{20em} | p{20em} | } 
\hline
 &  \\
\textbf{Grammar constructs} & \textbf{Ontological constructs}  \\
 &  \\
\hline

Convention Set & Style Guide \\ \hline
Contexts (Convention Set) & partly Conventions (Style Guide) \\ \hline
Context & Ignored Constructs (Context) \\ \hline
Conventions (Context) & partly Conventions (Style Guide) \\ \hline
Convention & Convention \\ \hline
Description property (Convention) & Description (Convention) \\ \hline
Pattern Descriptor & Context \\ \hline
Node Descriptor & partly Existence Constraint \\ \hline
Node Relation & partly Existence Constraint \\ \hline

Literal Expression & Literal Value \\ \hline
Variable Expression & Construct \\ \hline
Not Expression & Negation Constraint Combinator \\ \hline
And Expression & Conjunction Constraint Combinator \\ \hline
Or Expression & Disjunction Constraint Combinator \\ \hline
Comparison Expression & Comparison Constraint \\ \hline
Match Expression & Textual Form Constraint \\ \hline
In Expression & Set Membership Constraint \\ \hline
Is Expression & Type Constraint \\ \hline
Call Expression & Property (Construct) \\ \hline
Node Query Expression & partly Existence Constraint \\ \hline

Css Pattern & Pattern \\ \hline
Nodes (Css Pattern) & Number of Constructs (Pattern) \\ \hline
StylesheetNode & Stylesheet \\ \hline
Violations Log & Violations Log \\ \hline
Violations (Violations Log) & Violations (Violation Log) \\ \hline
Violation & Violation \\ \hline
Description (Violation) & Description (Violation) \\ \hline
Position (Violation) & Position in Stylesheet (Violation) \\ \hline
\end{longtable}
\end{center}
\caption{
	\label{tab:interpretation}
	Interpretation mapping}
\end{table}

\textbf{Ontological Classes and Subclasses}

The ontological concept of Style Guide is represented in the modeling grammar
as a Convention Set. Similarly, the notion of Convention maps to the
Convention class and the concept of Context maps to the Pattern Descriptor
class.

The ontological concept of Constraint appears in the modeling grammar as a
number of Expressions. These include Expressions used in mappings for the
subclasses of Constraint: Comparison Expression, Is Expression, Match
Expression, In Expression and Node Query Expression. Additionally, the
Constraint is mapped to And, Not and Or Expressions. These constructs cannot
serve as Constraints themselves but are used to form more complex Constraints.

The mappings for most of the Constraint subclasses are straight forward. For
example, the Comparison Constraint is represented as a Comparison Expression
and the Type Constraint appears as an Is Expression. Similarly, the Textual
Form Constraint is a Match Expression and the Set Membership Constraint maps
to the In Expression. That said, the Existence Constraint require further
attention. As defined in the ontology, the Existence Constraint requires the
subject to exist. In the modeling grammar this concept is indirectly presented
using multiple structures. Nodes in a pattern are specified using two ways:
either through using Node Descriptor and Node Relations, or through Node Query
Expression. The modeling grammar describes the target nodes from the pattern
and their parent nodes using the former approach, and nodes that are nested in
the target nodes --- using the latter. For example, when the pattern that
needs to be matched is a ruleset which contains a z-index property and appears
in a media query, the existence of the media query and the ruleset (and their
relationship) is described using Node Descriptors and a Node Relation. The
existence of the property contained in the ruleset is denoted using a Node
Query Expression.

The ontological constructs of Violation Log and Violation are mapped to the
Violation Log and Violation classes, respectively. The notion of a Stylesheet
is represented by the Stylesheet Node in the solution and a Pattern is mapped
to Css Pattern. The concept of a Construct is represented in the abstract
syntax tree as a Variable Expression. The reason behind this mapping is that
in the modeling grammar Variable Expression denote only Constructs.

\textbf{Ontological Properties}

The majority of the ontological properties are mapped to a single construct in
the solution. For example, the Description property of Convention class in the
ontology is directly mapped to the Description property of the Convention
class in the modeling grammar. However, some of the property mappings are not
that obvious. For example, the ontological property Conventions of class Style
Guide is represented by a combination of properties in the solution: Contexts
of Convention Set and Conventions of Context. Also, the notion of Ignored
Constructs maps to the Context class in the solution. The reason for these
mappings comes from the fact that the solution groups together conventions
with identical ignored constructs. As stated in the ontology description, each
convention specifies a set of constructs that need to be ignored while
searching for its violations. For example, while evaluating the constraints of
conventions related to newlines, the indentation constructs are typically
ignored. Because the ignored constructs are similar for most of the
conventions, the modeling grammar groups conventions that use the same ignored
constructs into contexts. Thus, the Convention Set contains Contexts and a
Context specifies the ignored constructs of all convention that it contains.

The Subject and the Value of a Constraint are mapped to the operands of
Expression. The subclasses of Constraint are based on the different values of
the Requirement property. Similarly, the Requirement of Constraint is
expressed through child classes of Expression.

The ontological property Violations of the Violation Log construct appears in
the modeling grammar as property Violations of the Violation Log class. The
Violation class exposes Description and Position properties that map to the
ontological properties Description and Position in Stylesheet. The Checked
property of Stylesheet does not have a representation in the system. The
property Property of Construct is denoted in the abstract syntax using a Call
Expression.

\subsection{Ontological Evaluation of the System}

The primary purpose of the representation and interpretation mappings between
the ontology and the modeling grammar is discovering discrepancies between the
two entities. The four types of ontological anomalies are considered in the
following subsections.

\subsubsection{Redundant Constructs} 

Construct redundancy is a type of discrepancy in which more than one modeling
construct can represent a single ontological construct. The representation
mapping illustrates that there are three candidates for this anomaly.

First, the Conventions property of Style Guide is matched to Contexts property
of Convention Set and the Conventions property of Context. Such mapping is
required since the modeling grammar groups Conventions that share the same
Ignored Constructs in a Context. Thus, a Style Guide in the modeling grammar
does not possess conventions but a number of Contexts that, in turn, contain
conventions. In this sense, the properties Contexts and Conventions together
represent the concept of Conventions.

Note that redundancy occurs when the two grammar constructs can independently
represent the same ontological concept. In the specific case, however, none of
the constructs taken individually can express the property. It is their
combination that represents the concept. Mapping a single ontological concept
to a combination of modeling grammar constructs is an accepted approach and
has been used in multiple studies~\cite{gehlert2007toward}. Thus, the two
constructs are not considered redundant.

Second, the Comparison, Is, In, Match, Node Query, And, Or and Not Expressions
map to the concept of Constraint. As defined in the ontology, a Constraint is
a class that has multiple subclasses depending on the specific requirement
they enforce. In this sense, Expressions that map to any subclasses of
Constraint, also map the Constraint class. These are the Comparison, Is, In,
Match and Node Query Expressions. The remaining part of Expressions are And,
Or and Not. These Expressions are used to combine Constraints into complex
Constraints. For example, a Context may contain a number of Constraints that
need to be put together using the And Expression. However, the And Expression
itself cannot be a Constraint --- it is used to build Constraints. In this
sense, none of the three Expressions can independently represent a Constraint.
Their combination with the Expressions mapped to the subclasses of Constraint
expresses the concept of Constraint. Thus, the expressions are not considered
redundant.

Third, the Existence Constraint is mapped to the Node Descriptor, Node
Relation and Node Query Expressions. Existence Constraint is a type of
Constraint that requires the Subject to exist. For example, the convention
\textit{Use a fallback property for RGBA and HSLA values} requires the
fallback declaration to have the same property as the matched declaration and
the value of the fallback declaration to be different than RGBA or HSLA.
However, before these, the convention requires the fallback declaration to
exist.

In the modeling grammar, when a Pattern Descriptor refers to multiple nodes,
the presence of a Node Descriptor denotes the existence of CSS Node and the
Node Relation specifies how the Nodes are connected in the Pattern Descriptor.
As described in the domain analysis section, each Pattern Descriptor has one
of more target nodes. These are the specific nodes that need to be iterated. While
the Node Descriptors and Node Relations describe the target nodes in the
Pattern and their parent nodes, the Node Query Expression is used to describe
parts of the pattern that are nested in the targets.

As with the previous two cases, the ontological concept cannot be represented
by Node Descriptors and Node Relations or by Node Query Expressions taken
individually. Their combined use allows expressing patterns. For this reason,
the constructs are not considered redundant.


\subsubsection{Construct Overload}

Construct overload emerges when a single modeling construct can represent a
number of ontological constructs. In the interpretation and representation
mappings every language construct is assigned a single ontological construct.
In this sense, there are not candidates for overload discrepancy.

\subsubsection{Construct Excess}

Construct excess is a discrepancy in which a modeling construct does not have
a mapping to an ontological construct. In the interpretation mapping all
constructs that belong to the abstract syntax are mapped to an ontological
construct. Thus, there are not candidates for construct excess from the
abstract syntax.

\subsubsection{Construct Deficit}

Construct deficit appears when an ontological construct does not have a
corresponding modeling structure. A candidate for such discrepancy is the
property Checked of class Stylesheet as it appears without a matching
construct in the system. However, maintaining the status of a Stylesheet is
considered outside the scope of the system. Such functionality is highly
dependent on the specific information offered by the IDE or text editor.
Support for this property remains in the environment of the system. In this
sense, the candidate is not considered an occurrence of deficit.


\subsubsection{Results}

In this section are presented the interpretation and representation mappings
between the modeling grammar and the designed domain-specific ontology.
Further, all types of discrepancies are analyzed and argumentation is provided
why they should not be considered anomalies. Having the analysis in mind, it
can be concluded that the ontological analysis provides sufficient evidence to
consider the designed system both ontologically clear and complete.

% !TEX root = thesis.tex
\chapter{Conclusion}
\label{sec:conclusion}

The variability of CSS conventions used in practice cannot be handled by
existing tools. Thus, developers often need to make sure their code complies
to a given style guide manually. The thesis offers a solution to this problem.
Its contribution is threefold:

\begin{enumerate}
	\item First, the need for CSS conventions is evaluated through analyzing a total of 1,987,139 public repositories. Results indicate that 58\% of all commits that maintain any form of CSS, still maintain plain CSS. The gathered data provides evidence to conclude that despite of the popularity of preprocessors, CSS is still handcrafted on GitHub in the beginning of 2015.
	\item Second, to discover existing CSS code conventions two search engines were used and the first 100 results of each search were analyzed. As a result, 28 style guides containing 471 conventions were discovered. Analysis indicates that 151 of the conventions are unique. A list containing the description of conventions, their sources and detailed analysis is presented. 
	\item Third, a domain-specific language that is capable of expressing the gathered conventions is proposed. A proof of concept consisting of two parts is developed: a standalone Python package and a plug-in for Sublime Text editor. The implementation illustrates that the suggested approach enables
	automatic detection of violations of CssCoco conventions. The designed language is validated using ontological analysis. A domain-specific ontology, based on Bunge-Wand-Weber top level ontology, is defined and used as a reference in the ontological analysis. The conducted analysis of ontological discrepancies indicates that the language is both ontologically clear and complete.
\end{enumerate}

% The conclusion is now half page and contains a summary of the results of each chapter

{
\bibliographystyle{ieeetr}
\bibliography{thesis}
}

\begin{appendices}
% !TEX root = thesis.tex
\chapter{Concrete Syntax of CssCoco}

This appendix contains all rules presented in the Context Syntax section of the Expressing CSS Code Conventions chapter. The following two snippets present the lexer and parser rules, respectively.

\begin{snippet}
\begin{verbatim}

Letter : [a-zA-Z] ;

EscapeSequence : "\\" "'" ;

Digit : ZeroDigit | NonZeroDigit ;

NonZeroDigit : [1-9] ;

ZeroDigit : [0] ;

Boolean : 'true' | 'True' | 'false' | 'False' ;

Identifier : (Letter)(Letter|Digit|'_'|'-')* ;

String : "'" (EscapeSequence | ~['])*? "'" ;

Integer : (ZeroDigit | NonZeroDigit Digit*) ;

Decimal : ( NonZeroDigit Digit* | ZeroDigit? ) '.' Digit+ ;

\end{verbatim}
\end{snippet}

The parser rules of CssCoco are presented in the next snippet.

\begin{snippet}
\begin{verbatim}
stylesheet : context* ;

context : Identifier ignore_clause? '{' convention* '}' ;

ignore_clause : 'ignore' (node_descriptor)+ (',' (node_descriptor)+)* ;

convention : 'forbid' pattern 'message' String
           | 'find' pattern ('where' logic_expr)? ('require'|'forbid') logic_expr 'message' String
           ;

pattern : node_declaration (('in'|'next-to') node_declaration)*
        | fork ('in' node_declaration)*
        ;

fork : '(' node_declaration (',' node_declaration)+ ')' ;

node_declaration : (Identifier '=')? semantic_node ;

node_descriptor : 'unique'? node_type ('{' (logic_expr|repeater) '}')? ;

repeater : Integer ',' Integer? | (',')? Integer ;

logic_expr : '(' logic_expr ')'
           | 'not' logic_expr
           | logic_expr 'and' logic_expr
           | logic_expr 'or' logic_expr
           | type_expr
           | arithmetic_expr
           ;

type_expr : arithmetic_expr operator='is' Identifier
          | node_descriptor+ ('before' | 'after') type_operand
          | node_descriptor+ 'between' type_operand 'and' type_operand
          ;

type_operand : Identifier | semantic_node ;

arithmetic_expr : ('-'|'+') arithmetic_expr
                | arithmetic_expr ('<'|'>'|'<='|'>='|'=='|'!=') arithmetic_expr
                | arithmetic_expr ('in'|'not in'|'match'|'not match') arithmetic_expr
                | call_expression
                | element
                ;
element : Boolean | Decimal | Integer | String | list_ ;

call_expression : call_expression '.' call_expression
                | Identifier ('(' (element | semantic_node ) ')')? 
                ;

list_ : '[' list_element (',' list_element)* ']' ;

list_element : Integer | Decimal | String | semantic_node ;

node_type : '(' node_type ')'
          | 'not' node_type
          | node_type 'and' node_type
          | node_type 'or' node_type
          | Identifier
          ;
\end{verbatim}
\end{snippet}


\end{appendices}

\end{document}
