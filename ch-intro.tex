\chapter{Introduction}

Code conventions put constraints on how code should be written in the context of a project,
organization or programming language. Style guides can comprise conventions that refer to
whitespacing, indentation, code layout, preference of syntactic structures, code patterns and anti-
patterns. They are mainly used to achieve code consistency, which in turn improves the readability,
understandability and maintainability of the code~\cite{Buse2010} ~\cite{buse2010learning}
~\cite{tenny1988program}.

Style guides are often designed in an ad hoc manner. Coding conventions typically live in documents
that contain a description of each rule in natural language accompanied by code examples. This is
the case with the style guidelines of Mozilla~\cite{Mozilla}, Google~\cite{Google},
GitHub~\cite{GitHub}, WordPress~\cite{Wordpress} and Drupal~\cite{Drupal}. It is the responsibility
of the developers to ensure that their code complies to a given style guide. Typically, they need to
read and understand the conventions and then apply them manually. Such an approach introduces a
number of issues. First, using natural language can make guidelines incorrect, ambiguous, implicit
or too general. Second, the fact that developers apply conventions manually increases the chances of
introducing violations involuntarily. There are a number of tools that can automatically detect
violations of conventions, however, current solutions are often hard to customize or are limited to
one type of violations, e.g. only whitespacing.

The core idea behind the project is to provide a solution that lets developers express an arbitrary
set of coding conventions and detect their violations automatically. Writing conventions
in an executable form could assist authors in detecting incorrect, ambiguous or inconsistent
guidelines. Automatic detection of violations could minimize the effort required by developers to
write code that complies to the guidelines. To meet the constraints of a Master’s project, the
implementation is limited to the domain of Cascading Style Sheets (CSS). The project requires
determining the need for CSS code conventions in organizations, collecting and analyzing available
style guides, and providing a way to express conventions. Specifically, the project attempts to
answer the following set of questions:

  \begin{description}
    \item \textbf{Research Question 1} Do developers still maintain plain CSS?
    \item \textbf{Research Question 2} What code conventions for CSS exist?
    \item \textbf{Research Question 3} How to express existing CSS code conventions?
  \end{description}

The thesis is organized as follows. Chapter \ref{sec:background} provides information about previous
studies and defines concepts and terms used throughout the thesis. Chapter \ref{sec:evaluating}
presents the research approach used to determine whether CSS is handcrafted and analysis of the
gathered results. Chapter \ref{sec:discovering} contains the research method  used to discover
existing code conventions and the results of the research. The design and validation of the DSL are
presented in Chapter \ref{sec:expressing}. Chapter \ref{sec:conclusion} concludes the thesis.

