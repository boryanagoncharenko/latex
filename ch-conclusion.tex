% !TEX root = thesis.tex
\chapter{Conclusion}
\label{sec:conclusion}

The variability of CSS conventions used in practice cannot be handled by
existing tools. Thus, developers often need to make sure their code complies
to a given style guide manually. The thesis offers a solution to this problem.
Its contribution is threefold:

\begin{enumerate}
	\item First, the need for CSS conventions is evaluated through analyzing a total of 1,954,102 public repositories. Results indicate that 58\% of all commits that maintain any form of CSS, still maintain plain CSS. The gathered data provides evidence to conclude that despite of the popularity of preprocessors, CSS is still handcrafted on GitHub in the beginning of 2015.
	\item Second, to discover existing CSS code conventions two search engines were used and the first 100 results of each search were analyzed. As a result, 28 style guides containing 471 conventions were discovered. Analysis indicates that 155 of the conventions are unique. A list containing the description of conventions, their sources and detailed analysis is presented. 
	\item Third, a domain-specific language that is capable of expressing the gathered conventions is proposed. A proof of concept consisting of two parts is developed: a standalone Python package and a plug-in for Sublime Text editor. The implementation illustrates that the suggested approach enables automatic detection of violations of CssCoco conventions. The designed language is validated using ontological analysis. A domain-specific ontology, based on Bunge-Wand-Weber top level ontology, is defined and used as a reference in the ontological analysis. The conducted analysis of ontological discrepancies indicates that the language is both ontologically clear and complete.
\end{enumerate}

% The conclusion is now half page and contains a summary of the results of each chapter